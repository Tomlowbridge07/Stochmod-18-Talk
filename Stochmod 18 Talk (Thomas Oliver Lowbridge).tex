\documentclass[11pt]{beamer}

\usetheme{Warsaw}
%\addtobeamertemplate{navigation symbols}{}{%
%    \usebeamerfont{footline}%
%    \usebeamercolor[fg]{footline}%
%    \hspace{1em}%
%    \insertframenumber/\inserttotalframenumber
%}

\beamertemplatenavigationsymbolsempty

\setbeamertemplate{footline}
{
  \leavevmode%
  \hbox{%
  \begin{beamercolorbox}[wd=.333333\paperwidth,ht=2.25ex,dp=1ex,center]{author in head/foot}%
    \usebeamerfont{author in head/foot} \hyperlink{Patrolling games}{\inserttitle}
  \end{beamercolorbox}%
  \begin{beamercolorbox}[wd=.333333\paperwidth,ht=2.25ex,dp=1ex,center]{title in head/foot}%
    \usebeamerfont{title in head/foot}\insertauthor
  \end{beamercolorbox}%
  \begin{beamercolorbox}[wd=.333333\paperwidth,ht=2.25ex,dp=1ex,right]{date in head/foot}%
    \usebeamerfont{date in head/foot}\insertshortdate{}\hspace*{2em}
    \insertframenumber{} / \inserttotalframenumber\hspace*{2ex} 
  \end{beamercolorbox}}%
  \vskip0pt%
}
\setbeamertemplate{headline}{}
\usefonttheme[onlymath]{serif}

\usepackage{color}
\usepackage{xcolor}
\usepackage{tikz}
\usepackage{amsmath}
\usepackage{amssymb}
\usepackage{amsthm}
\usepackage{amsfonts}
\usepackage{graphicx}
\usepackage{mathtools}
\usepackage{wrapfig}
\usepackage{multirow}
\usepackage{comment}
\usepackage{natbib}
\usepackage{appendix}
\usepackage[utf8]{inputenc}
%\usepackage{floatrow}
%\usepackage{newfloat}
\usepackage{subcaption}
\usepackage{bm}
\usepackage{hyperref}
\usepackage{tcolorbox}

\usetikzlibrary{calc}
\usetikzlibrary{fit}
\usetikzlibrary{shapes.misc,calc, positioning, hobby, backgrounds}

\tikzset{cross/.style={cross out, draw=black, minimum size=2*(#1-\pgflinewidth), inner sep=0pt, outer sep=0pt},
%default radius will be 1pt. 
cross/.default={1pt}}


%\DeclarePairedDelimiter{\floor}{\lfloor}{\rightfloor}
%\DeclarePairedDelimiter{\ceil}{\lceil}{\rceil}

\newcommand{\halflength}{\ensuremath{\floor{\frac{m}{2}}}}
\newcommand{\floor}[1]{\left \lfloor #1 \right \rfloor}
\newcommand{\ceil}[1]{\left \lceil #1 \right \rceil}
\newcommand{\pospart}[1]{\left( #1 \right)_{+}}
\newcommand{\negpart}[1]{\left( #1 \right)_{-}}
\newcommand{\set}[2]{\left\{ #1 \, | \, #2 \right\}}

\newcommand{\oneline}[1]{\resizebox{\dimexpr\paperwidth - 3ex}{!}{#1}}

%Text box tight style
\tcbset{mytight/.style={hbox,left=1mm,right=1mm,top=1mm,bottom=1mm,nobeforeafter}}

%\DeclareFloatingEnvironment[fileext=los,
 %   listname={List of Example Figures},
  %  name=Example Figure,
   % placement=tbhp,
    %within=section,]{examplefigure}

\author{Thomas Lowbridge and David Hodge}
\title{A Graph Patrol Problem with Locally-Observable Random Attackers}
%\setbeamercovered{transparent} 
%\setbeamertemplate{navigation symbols}{} 
%\logo{} 
\institute{University Of Nottingham,UK} 
\date{June 15, 2018} 
%\subject{} 
\begin{document}

\hypertarget{Patrolling games}{}
\begin{frame}
\titlepage
\end{frame}

%\begin{frame}
%\tableofcontents
%\end{frame}

\begin{frame}{Outline}

\begin{itemize}
\item Literature review
 \begin{itemize}
 \item Introduction to Game
  \begin{itemize}
  \item \hyperlink{Introduction to game: Pure game}{Pure game}
  \item \hyperlink{Introduction to game: Mixed game}{Mixed game}
  \end{itemize}
 \item Solved Graphs
  \begin{itemize}
  \item \hyperlink{Solved graphs: Hamiltonian graphs}{Hamiltonian graphs}
  %\item \hyperlink{Solved graphs: Complete bipartite graphs}{Complete bipartite graph}
  %\item \hyperlink{Solved graphs: Star graph}{Star graph}
  \item \hyperlink{Solved graphs: Line graph}{Line graph}
  \end{itemize}   
 \end{itemize}
\item \hyperlink{Problem with diametric strategy}{Problem with line graph strategy}
\item \hyperlink{Correction of diametric line graph strategy}{Correction of line graph strategy}
\item \hyperlink{Extension of correction strategy}{Extension of correction strategy}
%\item \hyperlink{Introduction to the elongated star}{Introduction to the elongated star}
\item \hyperlink{Future work}{Future work}
\end{itemize}
\end{frame}

\section[]{Game with non-observed}
\hypertarget{Introduction to game: Pure game}{}
\begin{frame}{\insertsection}

A Patrolling game with random attackers, \textcolor{purple}{$G=G(Q,\bm{X},\bm{\lambda},\bm{c})$} is made of 4 major components
\begin{itemize}
\item A \textcolor{purple}{Graph, $Q=(N,E)$}, made of nodes, $N$ ($|N|=n$), and a set of edges, $E$.
\item A vector of \textcolor{purple}{attack times, $\bm{X}=(X_{1},...,X_{n})$} .
\item A vector of \textcolor{purple}{arrival rates, $\bm{\lambda}=(\lambda_{1},...,\lambda_{n})$}.
\item A vector of \textcolor{purple}{costs, $\bm{c}=(c_{1},...,c_{n})$}
\end{itemize}

\pause

The game is played over an infinite time horizon, $\mathcal{T}={0,1,....}$

A patroller's policy in the game is a walk (with waiting) on the graph,
$$W: \mathcal{T} \rightarrow N$$
A deterministic, stationary policy, $\pi$ is a subset of all the possible walks, we will focus on these types of policies [\textcolor{red}{DO I NEED A REASON ???}] 

\end{frame}


\end{document}
